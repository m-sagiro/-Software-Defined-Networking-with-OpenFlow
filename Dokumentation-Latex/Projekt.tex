%*************************************************************************
% Dokument Einstellungen
%*************************************************************************
\documentclass[fontsize=12pt,paper=a4,open=any,parskip=half,
  twoside=false,toc=listof,toc=bibliography,fleqn,leqno,
  captions=nooneline,captions=tableabove,british]{scrbook}
%*************************************************************************
% Importieren von Paketen die benutzt werden
%*************************************************************************
\usepackage[utf8]{inputenc} % load early
\usepackage[T1]{fontenc}    % load early
\usepackage[ngerman]{babel}
\usepackage[autostyle=true]{csquotes}
\usepackage{graphicx, booktabs, float, scrhack}
\usepackage{caption}
\usepackage{listings}
%\usepackage{fancyref}
%\usepackage{showkeys}
\usepackage[svgnames]{xcolor}
\usepackage{amsmath,amssymb}
\usepackage[automark]{scrlayer-scrpage}
\usepackage[backend=biber,style=alphabetic]{biblatex} %,sortcase=false,

%*************************************************************************
% Bibliographies - Zitatquellen
%*************************************************************************
\addbibresource{Projekt.bib}

%*************************************************************************
% Weitere Dokument Einstellungen
%*************************************************************************
\PassOptionsToPackage{hyphens}{url} 
\usepackage[hidelinks]{hyperref}  % load late
\setkomafont{disposition}{\sffamily}
%*************************************************************************
% Dokumentanfang
%*************************************************************************
\begin{document}
%Aktivierung römische Seitenzahlen
\frontmatter

%Titelblatt Einstellungen
\titlehead{% siehe KOMA-Script-Anleitung
  \begin{minipage}[t]{0.65\textwidth}
    \raggedright
    			Frankfurt University of Applied Sciences\\
				Fachbereich 2: Informatik und Ingenieurwissenschaften\\
				Studiengang: Informatik (B.Sc.)\\
  \end{minipage}
  \hfill
  \raisebox{-\dimexpr\totalheight-\ht\strutbox\relax}{
    \includegraphics[width=5cm]{Bilder/fra-uas}
  }
}

\subject{Projektarbeit}
\title{Software-defined Networking mit Openflow}
\subtitle{}
\author{Mücahit Sagiroglu\\
Matrikelnummer: 1228852\\
James Belmonte\\
Matrikelnummer: 1340604\\
Naghmeh Ghavidel Rostami\\
Matrikelnummer: 1249307\\
Tung Trinh\\
Matrikelnummer:\\
}
\date{Vorgelegt am: 27. Januar 2022}
\publishers{Dozent: Maurizio Petrozziello\\
Modul 25: Informatik Projekt\\
Software-defined Networking mit Openflow\\
Wintersemester 2021/2022\\
}

\maketitle
%Eigenständigkeitserklärung
\chapter*{Eigenständigkeitserklärung}
Hiermit erklären wir, dass wir die vorliegende Arbeit eigenständig verfasst, keine anderen als die
angegebenen Quellen und Hilfsmittel verwendet sowie die aus fremden Quellen direkt oder indirekt
übernommenen Stellen/Gedanken als solche kenntlich gemacht haben. Diese Arbeit wurde noch keiner
anderen Prüfungskommission in dieser oder einer ähnlichen Form vorgelegt. Sie wurde bisher auch nicht
veröffentlicht.

Hiermit stimmen wir zu, dass die vorliegende Arbeit von der Prüferin/ dem Prüfer in elektronischer Form
mit entsprechender Software auf Plagiate überprüft wird.

\begin{figure}[H]
	\centering
	\includegraphics[width=1\linewidth]{Bilder/unterschrift}
\end{figure}

%Inhaltsverzeichnes & Abbildungsverzeichnis & Tabellenverzeichnis
\tableofcontents
\listoffigures
\listoftables
\lstlistoflistings

%Aktivierung arabische Seitenzahlen
\mainmatter % Seite fängt mit 1 an



%Kapitel: Einleitung
\chapter{Einleitung}\label{ch:intro}
Seit der Einführung des ... existiert das ..., welcher, wie der Name ausdrückt, einige ... für .... zur Verfügung stellt.

\section{Software-defined Networking}\label{sdn}
asd
\subsection{Einleitung von James}\label{einl-james}
asddsa
\subsection{Einleitung von Naghmeh}\label{einl-naghmeh}
asddsa
\subsection{Einleitung von Tung}\label{einl-tung}
asddsa
\subsection{Einleitung von Mücahit}\label{einl-müco}
asddsa

\section{Motivation}
Was hat uns zum schreiben der Projektarbeit (in Bezug auf die Problemstellung) gebracht?

https://www.scribbr.de/aufbau-und-gliederung/motivation-bachelorarbeit/
\section{Problemstellung}
Die Problemstellung beschreibt das Forschungsproblem, das du mit deiner Abschlussarbeit lösen möchtest.
Was ist das Thema des Projekts und wie lautet die konkrete Fragestellung?

https://www.scribbr.de/anfang-abschlussarbeit/problemstellung/
\section{Zielsetzung}
Das hier würde wegfallen. Ersetzt durch Projektziel.

Die Zielsetzung deiner Bachelorarbeit sollte deinen Lesenden einen Einblick in das „Warum“ und das „Wie“ deiner Untersuchung geben.

Warum führst du die Forschung durch und wie wirst du dieses Ziel erreichen?

https://www.scribbr.de/anfang-abschlussarbeit/zielsetzung-formulieren/
\section{Aufbau der Arbeit}
Wie ist die restliche Projektarbeit aufgebaut? Was kommt noch?


\chapter{Projekt}
asd
\section{Projektziel}
Wie lautet das Ziel des Projekts?
\section{Projektumfeld}
\subsection{Mininet}
Installation von Tung...
\subsection{Floodlight}
Installation von Tung...
\section{Vorgehen}
Wie sieht die Vorgehensweise aus?


\chapter{Projektplanung}
asd
\section{Aufbau des Projektplanes}
asd
\section{Festlegen von Meilensteinen}
Welche Meilensteine wurden festgelegt?


\chapter{Durchführung des Projektes}
Projektdurchführung

\chapter{Gesamtergebnis}
Analyse der Ergebnisse
Kritische Betrachtung


\chapter{Fazit}\label{ch:fazit}
Eventuell als Überpunkt zu Gesamtergebnis?

Was ist der finale Stand des Projekts?
Inwiefern wurden die Ziele erreicht?
Wie sehen eventuelle Prognosen für die Zukunft aus?
Inwiefern können die Ergebnisse des Projekts weiter genutzt werden?
\par























%Kapitel: Java.Util
\chapter{Kapitel 1}\label{ch:j.u}
Hier kommt Kapitel 1. Aufzählungen gehen so:
\begin{itemize}
 \item Aufzählung 1
 \item Aufzählung 2
 \item Aufzählung 3
 \item ...
\end{itemize}
Hier kann der Text weitergehen.

%Kapitel: Interface
\chapter{Kapitel 2}\label{ch:interfaces}
Hier steht Kapitel 2. Hier kommt ein Listing:
\begin{lstlisting}[language=Java,
					caption={Deklaration eines Interfaces},
					backgroundcolor = \color{lightgray},
					captionpos=b,
					numbers=left,
					keywordstyle=\color{RoyalBlue},
    				rulecolor=\color{black},
   		 			upquote=true, 
					showstringspaces=false,
    				breaklines=true,
    				frame=single,
					aboveskip=2em,
					label={interface-deklaration},
]
public interface Interface1 extends Interface2, Interface3 {
	...
	public int methode(int zahl1, int zahl2);
	...
}
\end{lstlisting}
\captionsetup{justification=centering,margin=2cm}

Hier geht der Text weiter. Und so bindet man ein Figure ein(Bild im Ordner Bilder zu finden):


\begin{figure}[htbp]
 \centering
 \includegraphics[width=0.6\textwidth]{Bilder/bildname}
 \captionsetup{justification=centering,margin=2cm}
 \caption{Beziehungen von Klassen und Interfaces \autocite{jtpinterface}}
 \label{interface-relation}
\end{figure}

Hier kann der Text weitergehen.


\section{Unterkapitel 1}\label{sec:c.f}
Hier ist ein Unterkapitel (Section). Hier paar Aufzählungen:

\begin{itemize}
 \item public boolean add(E e) 
 \item public boolean remove(Object element)
 \item public int size()
 \item public boolean contains(Object element)
 \item public boolean isEmpty()
\end{itemize}

Text geht weiter..... Hier kommt eine Tabelle:

\begin{table}[htbp]
\caption{Eigenschaften von Vector, PriorityQueue und HashSet}
\label{data-table}
\centering
  \begin{tabular}{l  c  c  c} 
\toprule
    Eigenschaften & Vector & PriorityQueue & HashSet\\ 
\midrule  
    Doppelte Einträge erlaubt:   		& Ja  	&  Ja  	& Nein \\
    Reihenfolge:   						& Ja 	&  Ja 		& Nein \\ 
	Veränderbar:   						& Ja	&  Nein 	& Ja \\ 
	Thread-Safe:   						& Ja 	& Nein 	& Nein\\ 
  \end{tabular}

\end{table}


\section{Unterkapitel 2}\label{ch:vector}
Hier geht der Text weiter. Noch eine Tabelle:

\begin{table}[htbp]
\caption{Eigenschaften der Python Datenstrukturen \autocite{listuple}}
\label{python-data-table}
\centering
  \begin{tabular}{l  c  c  c c} 
\toprule
    Eigenschaften & List & Tuple & Set & Dict\\ 
\midrule  
    	Doppelte Einträge erlaubt:   			& Ja  	&  Ja  		& Nein 	& Keine doppelten Keys\\
    	Reihenfolge:   						& Ja 	&  Ja 		& Nein 	& Ja\\ 
	Veränderbar:   						& Ja	&  Nein 	& Ja 		& Ja\\ 
	Thread-Safe:   						& Ja 	& Ja 	& Ja 		& Ja\\ 
  \end{tabular}

\end{table}

%Kapitel: Fazit




\printbibliography[title=Literaturverzeichnis]

\end{document}