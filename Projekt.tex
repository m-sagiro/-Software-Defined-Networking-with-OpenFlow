%*************************************************************************
% Dokument Einstellungen
%*************************************************************************
\documentclass[fontsize=12pt,paper=a4,open=any,parskip=half,
  twoside=false,toc=listof,toc=bibliography,fleqn,leqno,
  captions=nooneline,captions=tableabove,british]{scrbook}
%*************************************************************************
% Importieren von Paketen die benutzt werden
%*************************************************************************
\usepackage[utf8]{inputenc} % load early
\usepackage[T1]{fontenc}    % load early
\usepackage[ngerman]{babel}
\usepackage[autostyle=true]{csquotes}
\usepackage{graphicx, booktabs, float, scrhack}
\usepackage{caption}
\usepackage{listings}
%\usepackage{fancyref}
%\usepackage{showkeys}
\usepackage[svgnames]{xcolor}
\usepackage{amsmath,amssymb}
\usepackage[automark]{scrlayer-scrpage}
\usepackage[backend=biber,style=alphabetic]{biblatex} %,sortcase=false,

%*************************************************************************
% Bibliographies - Zitatquellen
%*************************************************************************
\addbibresource{Projekt.bib}

%*************************************************************************
% Weitere Dokument Einstellungen
%*************************************************************************
\PassOptionsToPackage{hyphens}{url} 
\usepackage[hidelinks]{hyperref}  % load late
\setkomafont{disposition}{\sffamily}
%*************************************************************************
% Dokumentanfang
%*************************************************************************
\begin{document}
%Aktivierung römische Seitenzahlen
\frontmatter

%Titelblatt Einstellungen
\titlehead{% siehe KOMA-Script-Anleitung
  \begin{minipage}[t]{0.65\textwidth}
    \raggedright
    			Frankfurt University of Applied Sciences\\
				Fachbereich 2: Informatik und Ingenieurwissenschaften\\
				Studiengang: Informatik (B.Sc.)\\
  \end{minipage}
  \hfill
  \raisebox{-\dimexpr\totalheight-\ht\strutbox\relax}{
    \includegraphics[width=5cm]{fra-uas}
  }
}

\subject{Hausarbeit}
\title{Das Paket Java.Util}
\subtitle{Ein Überblick auf die Hilfsklassen von Java}
\author{Mücahit Sagiroglu\\
Matrikelnummer: 1228852\\
}
\date{Vorgelegt am: 1. Februar 2022}
\publishers{Dozent: Taufik Ghalayini\\
Modul 24: Aktuelle Themen der Informatik – Wahlpflicht-Seminar\\
Java-Pakete (JP): Übersicht und Beispiele\\
Wintersemester 2021/2022\\
}

\maketitle 


%Inhaltsverzeichnes & Abbildungsverzeichnis & Tabellenverzeichnis
\tableofcontents
\listoffigures
\listoftables
\lstlistoflistings

%Aktivierung arabische Seitenzahlen
\mainmatter % Seite fängt mit 1 an



%Kapitel: Einleitung
\chapter{Einleitung}\label{ch:intro}
Seit der Einführung des ... existiert das ..., welcher, wie der Name ausdrückt, einige ... für .... zur Verfügung stellt.

%Kapitel: Java.Util
\chapter{Kapitel 1}\label{ch:j.u}
Hier kommt Kapitel 1. Aufzählungen gehen so:
\begin{itemize}
 \item Aufzählung 1
 \item Aufzählung 2
 \item Aufzählung 3
 \item ...
\end{itemize}
Hier kann der Text weitergehen.

%Kapitel: Interface
\chapter{Kapitel 2}\label{ch:interfaces}
Hier steht Kapitel 2. Hier kommt ein Listing:
\begin{lstlisting}[language=Java,
					caption={Deklaration eines Interfaces},
					backgroundcolor = \color{lightgray},
					captionpos=b,
					numbers=left,
					keywordstyle=\color{RoyalBlue},
    				rulecolor=\color{black},
   		 			upquote=true, 
					showstringspaces=false,
    				breaklines=true,
    				frame=single,
					aboveskip=2em,
					label={interface-deklaration},
]
public interface Interface1 extends Interface2, Interface3 {
	...
	public int methode(int zahl1, int zahl2);
	...
}
\end{lstlisting}
\captionsetup{justification=centering,margin=2cm}

Hier geht der Text weiter. Und so bindet man ein Figure ein:

\begin{figure}[htbp]
 \centering
 \includegraphics[width=0.6\textwidth]{bildname}
 \captionsetup{justification=centering,margin=2cm}
 \caption{Beziehungen von Klassen und Interfaces \autocite{jtpinterface}}
 \label{interface-relation}
\end{figure}

Hier kann der Text weitergehen.


\section{Unterkapitel 1}\label{sec:c.f}
Hier ist ein Unterkapitel (Section). Hier paar Aufzählungen:

\begin{itemize}
 \item public boolean add(E e) 
 \item public boolean remove(Object element)
 \item public int size()
 \item public boolean contains(Object element)
 \item public boolean isEmpty()
\end{itemize}

Text geht weiter..... Hier kommt eine Tabelle:

\begin{table}[htbp]
\caption{Eigenschaften von Vector, PriorityQueue und HashSet}
\label{data-table}
\centering
  \begin{tabular}{l  c  c  c} 
\toprule
    Eigenschaften & Vector & PriorityQueue & HashSet\\ 
\midrule  
    Doppelte Einträge erlaubt:   		& Ja  	&  Ja  	& Nein \\
    Reihenfolge:   						& Ja 	&  Ja 		& Nein \\ 
	Veränderbar:   						& Ja	&  Nein 	& Ja \\ 
	Thread-Safe:   						& Ja 	& Nein 	& Nein\\ 
  \end{tabular}

\end{table}


\section{Unterkapitel 2}\label{ch:vector}
Hier geht der Text weiter. Noch eine Tabelle:

\begin{table}[htbp]
\caption{Eigenschaften der Python Datenstrukturen \autocite{listuple}}
\label{python-data-table}
\centering
  \begin{tabular}{l  c  c  c c} 
\toprule
    Eigenschaften & List & Tuple & Set & Dict\\ 
\midrule  
    	Doppelte Einträge erlaubt:   			& Ja  	&  Ja  		& Nein 	& Keine doppelten Keys\\
    	Reihenfolge:   						& Ja 	&  Ja 		& Nein 	& Ja\\ 
	Veränderbar:   						& Ja	&  Nein 	& Ja 		& Ja\\ 
	Thread-Safe:   						& Ja 	& Ja 	& Ja 		& Ja\\ 
  \end{tabular}

\end{table}

%Kapitel: Fazit
\chapter{Fazit}\label{ch:fazit}
Mit der Hausarbeit sollte eine Übersicht auf das .... gegeben und geschaut werden, ob gleiche oder ähnliche Architekturen in ..... vorhanden sind. Hierzu wurde der Bereich auf die .... und .... eingegrenzt, da das .... sehr umfangreich ist. \par



\printbibliography[title=Literaturverzeichnis]



% Die eidesstattliche Erklärung mit Unterschrift
\chapter*{Eigenständigkeitserklärung}

Hiermit versichere ich, dass ich die vorliegende Arbeit selbstständig und ohne Benutzung
anderer als der angegebenen Hilfsmittel angefertigt habe. Alle Stellen, die wörtlich oder
sinngemäß aus veröffentlichten und nicht veröffentlichten Schriften entnommen sind,
sind als solche kenntlich gemacht. Die Arbeit hat in gleicher Form noch keiner anderen
Prüfbehörde vorgelegen.

\vspace{2cm}

Offenbach am Main, den 01.02.2022 \hfill Mücahit Sagiroglu \hspace{2cm}

\end{document}